\documentclass{article}
\usepackage[utf8]{inputenc}
\usepackage[hmargin=6em]{geometry}
\usepackage{amsmath}
\usepackage{hyperref}
\setlength{\parskip}{0.6em}
 \hypersetup{
    colorlinks=true,
    linkcolor=blue,
    filecolor=magenta,      
    urlcolor=cyan,
    pdftitle={Overleaf Example},
    pdfpagemode=FullScreen,
    }
    \urlstyle{same}
 
\title{\textbf{Primeiro Trabalho - Relatório Parcial}\\
\large Computação Concorrente (MAB-117) -- 2021/1 Remoto\\
\textbf{\Large Paralelização de integração pelo método de Simpson}}
\author{Christopher Ciafrino, Igor Vilhalba}

\begin{document}

\maketitle

\section{Descrição do problema}
O \emph{método de Simpson} é um método de integração numérica
onde utilizamos um \emph{polinômio de Lagrange} $P(x)$ para aproximar uma função
$ f(x)$ em um subintervalo $[a,b]$ tal que:

\begin{equation}
    \int_a^b{f(x)\,dx} \approx \int_a^b{P(x)\,dx}
\end{equation}

$P(x)$ aproxima $f(x)$ utilizando três pontos
distintos: $(a, f(a))$, $(m, f(m))$ e $(b, f(b))$, onde $m = \frac{a + b}{2}$.\\
Isso nos dá a seguinte expressão para $P(x)$

\begin{equation}
    P(x) = f(a)\frac{(x - m)(x - b)}{(a - m)(a - b)}
    + f(m)\frac{(x - a)(x - b)}{(m - a)(m - b)}
    + f(b)\frac{(x - a)(x - m)}{(b - a)(b - m)}
\end{equation}

Dada essa aproximação, podemos encontrar integral de $P(x)$ no intervalo $[a,b]$

\begin{equation}\label{eq:simpson}
    \int_a^b{P(x)\,dx} = \frac{b - a}{6}
    \left[
    f(a) + 4f(m) + f(b)
    \right]
\end{equation}

Entretanto, essa aproximação só é boa para intervalos
pequenos. Uma solução para isso é: dada uma função $f(x)$
definida em um intervalo $[a,b]$, dividiremos o intervalo em
um número $n$ (par) de subintervalos $[x_i,x_{i+1}]$, para
$i = 0, 1, ..., n - 1$ e calcularemos a integral desses
subintervalos.

Ou seja, primeiramente precisamos calcular o valor de $f(x)$
em $n + 1$ pontos $x_k$, com $k = 0,1,...,n$, de tal forma que $x_k = a + kh$ e $h = \frac{a - b}{n}$
é o tamanho de cada um desses subintervalos.

Com essa abordagem, obtemos a seguinte nova igualdade:

\begin{equation}\label{eq:simpson_composta}
    \int_a^b{P(x)\,dx} = \frac{h}{3}
    \left[
        f(x_0) + 2\sum_{i=1}^{\frac{n}{2}-1}{f(x_{2i})}
        + 4\sum_{i=1}^{\frac{n}{2}}{f(x_{2i - 1}) + f(x_n)}
    \right]
\end{equation}

Dada essa descrição do método, a entrada esperada do
nosso programa é uma expressão matemática, que calculará
os valores para f(x), o limite inferior e superior do
intervalo de integração, a quantidade de subintervalos (que
deve ser par!) e a quantidade de threads que se deseja
executar. A saída é a soma obtida em cada thread.

Por exemplo, vamos considerar uma função qualquer $f(x)$.
No nosso programa, criaremos um vetor $vet$ (não
necessariamente com esse nome), de tamanho $n + 1$.\\
Cada elemento $vet[i]$ desse vetor deve corresponder a
$f(x_i)$, para $i = 0,1,...,n$

Como podemos ver em \eqref{eq:simpson_composta}, há dois
somatórios, que podem ser facilmente divididos. Além disso,
os resultados são independentes, o que torna possível dividir
essa tarefa entre múltiplos fluxos de execução.

\section{Projeto inicial da solução concorrente}

\subsection{Estratégia de paralelização}
Esse problema pode ser paralelizado de duas formas diferentes.
Para falarmos de cada uma, vamos chamar de
$\Sigma_{2i}$ e $\Sigma_{2i-1}$, respectivamente, os somatórios
$\sum_{i=1}^{\frac{n}{2}-1}{f(x_{2i})}$ e
$\sum_{i=1}^{\frac{n}{2}}{f(x_{2i - 1})}$ que aparecem
em \eqref{eq:simpson_composta}.

\begin{enumerate}
    \item  Uma forma possível é dividir as tarefas em um número
    par de sub-tarefas. Isso porque dividiríamos cada
    um dos somatórios em um número igual fluxos de execução, onde
    cada fluxo calcularia exclusivamente ou $\Sigma_{2i}$, ou
    $\Sigma_{2i-1}$.\\
    Por exemplo, se o usuário desejar que a integral seja calculada
    utilizando $6$ threads, $3$ calcularão \emph{somente} o
    somatórios $\Sigma_{2i}$ e $3$ calcularão \emph{somente}
    $\Sigma_{2i-1}$, totalizando as $6$ threadssolicitada.
    \item  A outra forma consiste em calcular uma parte de
    $\Sigma_{2i}$ e de $\Sigma_{2i-1}$ em cada fluxo de
    execução. Nessa abordagem, o número de threads não fica
    restrito a um número par.\\
    Por exemplo, vamos supor que o usuário queira que a integral
    seja calculada utilizando $3$ threads, com $n$ subintervalos,
    e seja $m = n/2$.\\
    \begin{itemize}
        \item Uma thread calculará:
        \begin{itemize}
            \item $\Sigma_{2i}$ com $i=1,2,...\,,(m-1)/3$
            \item e $\Sigma_{2i-1}$ com $i=1,2,...\,,m/3$
        \end{itemize} 
        \item outra thread calculará:
        \begin{itemize}
            \item $\Sigma_{2i}$ com $i=\frac{m-1}{3}+1,\frac{m-1}{3}+2,...\,,(2m-2)/3$
            \item  e $\Sigma_{2i-1}$ com $i=(m/3)+1,(m/3)+2,...\,,2m/3$
        \end{itemize}
        \item e a thread restante calculará:
        \begin{itemize}
            \item $\Sigma_{2i}$ com $i=\frac{2m-2}{3}+1,\frac{2m-2}{3}+2,...\,,m-1$
            \item  e $\Sigma_{2i-1}$ com $i=(2m/3)+1,(2m/3)+2,...\,,m$
        \end{itemize}
    \end{itemize}
\end{enumerate}

Nós optamos pela segunda forma, já que o usuário terá
maior liberdade para escolher o número de fluxos de execução.
Além disso, acreditamos que essa estratégia
faça melhor uso da \emph{cache}, pois como um somatório utiliza
os índices ímpares e o outro os índices pares do vetor que guarda
os valores de $f(x)$, na prática esses índices posições intercaladas
na memória e pertencem à mesma porção da memória.
Ou seja, se uma porção do vetor for colocada na \emph{cache},
a thread aproveitará totalmente essa
porção que foi carregada, não só os índices pares ou ímpares.

\subsection{Planejamento de implementação}

%Para podermos implementar dessa maneira, precisaremos
%de uma \emph{struct} que guardará o resultado da porção
%de $\Sigma_{2i}$ e $\Sigma_{2i-1}$ calculada por cada fluxo
%de execução. Essa estrutura deve ser alocada dinamicamente
%por cada thread e retornada ao fluxo principal, que é onde
%serão somados os resultados obtidos.
Não precisaremos retornar nenhuma \emph{struct} de cada thread.
Basta retornarmos a soma dos resultados das porções de
$\Sigma_{2i}$ e $\Sigma_{2i-1}$ calculados nas threads.\\
Basta somarmos esses resultados na função que chama as
threads, então teremos o valor total da integral.

O vetor onde serão armazenados os valores de $f(x_i)$, para
$i = 0,1,...\,,n$ terá escopo global, assim como a variável
que armazenará quantas threads serão executadas. Dessa
forma, apenas precisaremos passar para as threads um 
\emph{id} para que saibamos que porção dos somatórios
precisamos calcular nela.

\section{Testes}

\subsection{Teste de corretude}
Para testar a corretude nós vamos escolher diversos tipos de funções, desde 
casos mais simples (integrais que sabemos na mão) até casos mais complexos 
(um exemplo seria uma integral sem primitiva analítica), e testá-los com
diversos intervalos (em torno de 10, com tamanhos distintos). Para determinar
se o valor encontrado está correto, usaremos a fórmula do erro do próprio método. Que é dada como: a integral calculada pelo
método deve possuir
erro inferior a

$$
\frac{M(b - a)h^4}{180}
$$

em que $M$ é o valor máximo de $f^{(4)}(x)$ no intervalo $[a, b]$ e $h$ é a quantidade de blocos em que dividimos o intervalo, ou seja, $h = \frac{(b - a)}{n}$.

\subsection{Teste de desempenho}
Na parte de desempenho teremos uma abordagem um pouco distinta, daremos mais ênfase em testar intervalo com magnitudes diferentes
em busca de variações na performance (intervalos muito grandes e muito pequenos, por exemplo). Vamos pegar todas funções testadas anteriormente e adicionar alguns casos extremos adicionais, para
ver como o programa concorrente se comporta.

\section{Referências Bibliográficas}

\begin{enumerate}
    \item \href{https://www.wikiwand.com/en/Simpson%27s_rule#/Simpson's_rules_in_the_case_of_narrow_peaks}{Simpson's Rule}
    \item Apostila de Cálculo Númerico do professor Collier (fornecida na disciplina)
    \item \href{https://github.com/zserge/expr}{expr -- biblioteca utilizada para interpretar expressões matemáticas}
\end{enumerate}

\end{document} 
